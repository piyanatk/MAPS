\documentclass[12pt]{article}
\usepackage[usenames,dvips]{color}

\textwidth=16.25cm
\textheight=22.5cm
\oddsidemargin=0.1cm
%\pagestyle{empty}
\topmargin=1.cm
\headsep=0.5cm


\newcommand{\as}[2]{\mbox{#1\farcs #2}}
\newcommand{\am}[2]{$#1^{'}\,\hspace{-1.7mm}.\hspace{.1mm}#2$}
\newcommand{\HI}{\mbox{H\,{\sc i}}}
\newcommand{\kms}{\mbox{\rm km\,s$^{-1}$}}
\newcommand{\SK}{$S^{3}$}
\def\lsim{~\rlap{$<$}{\lower 1.0ex\hbox{$\sim$}}}
\def\gsim{~\rlap{$>$}{\lower 1.0ex\hbox{$\sim$}}}

\begin{document}

\noindent{To: Haystack SKA Group}
\bigskip

\noindent{From: Lynn D. Matthews}
\bigskip

\noindent{Subject: Limitations and Bugs in the MAPS Software}
%\noindent{\color{red}Subject: Importing Model Skies into MAPS}
\bigskip

\noindent{Date: October 10, 2009}
\noindent{Updated: March 3, 2010}
\noindent{Updated: April 27, 2010}
\noindent{Updated: May 13, 2010}

\bigskip

\section{Background}
Below is a compilation of known bugs,
limitations, and other issues in 
the  MAPS Software package as of April 2010. 
All tests were performed with a version
of MAPS installed in November 2008. Problems are listed in no
particular order.  

\begin{flushleft}

\section{Limitations and Annoyances}


{\bf{\color{red} Problem 1:}} No comprehensive user manual exists for MAPS. 

{\bf Solution/Comments:} Create our own?
\bigskip

{\bf {\color{red}Problem 2:}} When running {\sf MAPS$\underline~$im2uv}, 
one must manually enter a scaling factor via the ``- n''
  switch  in
  order that the output data set will have units of ``Jy
  steradian$^{-1}$''. It is easy to make a mistake in this step,
  particularly since LOsim images require different handling from most
  other images.

{\bf Solution:} Modify
{\sf MAPS$\underline~$im2uv} to perform this scaling automatically.

\bigskip

{\bf {\color{red}Problem 3:}} {\sf MAPS$\underline~$im2uv} does not
retain or pass along 
any information from the header of an input FITS
image after it FFTs  it.
Consequently, when the data are passed to 
{\sf visgen},  {\sf visgen} does not have any {\it a priori}
information about the coordinates of the
field center or the field-of-view, and these values must
be explicitly passed to {\sf visgen} through the obs spec file. 
If the field-of-view (FOV) that the user specifies
in the ``obs spec'' file does not match that of the original input
image (including any zero padding that has been added to its
periphery), {\sf
visgen} will {\it not} select a sub-region of the input image; it
will assume that whatever the user has specified is the intrinsic 
FOV of the input model and rescale the spatial coordinates accordingly.


{\bf Solution/Comments:} Coordinate information should somehow be retained and passed
between MAPS modules. In case a user wishes to adopt
arbitrary coordinates or rescale an input image, an option 
to ignore or override this coordinate information might be desirable. 

\bigskip

{\bf {\color{red}Problem 4:}} {\sf LOsim} creates images with epoch 1950.0
coordinates. If one runs such an image through MAPS, the output epoch
of the newly generated data will be
J2000.0, but the coordinates will not be precessed.

{\bf Solution/Comments:} Tried adding an ``EPOCH" keyword to the header of a
LOsim-generated image, but this did not remedy the problem. 
MAPS modules need to be smarter in reading and passing
along coordinate information directly from input images. This will be
a non-issue of {\sf LOsim} is retired.

\bigskip

{\bf {\color{red}Problem 5:}} Images generated by {\sf LOsim} 
have RA increasing to the right; this is non-standard and 
can create later hassles and confusion.

{\bf Solution/Comments:} Either retire {\sf LOsim} or modify it 
to output images that are
transposed east-west.

\bigskip


{\bf {\color{red}Problem 6:}} If one attempts to ``observe'' a source from a particular 
observatory site at a time when it is not visible, 
{\sf visgen} will  produce an
output of all zeros, without any warning messages.

{\bf Solution/Comments:} {\sf visgen} should issue warnings if a source is not
visible. {\sf maps2uvfits} should also issue a warning if the input
file is all zeros. A tool for computing source elevation as a function of time
from an arbitrary observatory site would be valuable.

\bigskip

{\bf {\color{red}Problem 7:}} Often an externally-generated 
2-D FITS image will have additional axes that convey
e.g., STOKES and FREQUENCY information. {\sf MAPS$\underline~$im2uv}
chokes on these extra axes and will report the error message: 
``input image is NOT 2 dimensional. wrong wrong wrong.'' 

{\bf Solution/Comments:} Allow {\sf MAPS$\underline~$im2uv} 
to ignore all non-coordinate axes
provided that they are one-dimensional. Allowing {\sf
  MAPS$\underline~$im2uv} to operate on data cubes might also be
useful for some applications.

\bigskip

{\bf {\color{red}Problem 8:}} Aliasing will occur when imaging model 
visibility data 
created by {\sf visgen} in cases where a user-generated sky
model (FFTed via {\sf MAPS$\underline~$im2uv}) is not padded with
zeros first. 
Adding the padding requires  extra steps each time a model sky
is created.

{\bf Solution/Comments:} Add the capability to pad an input image to 
MAPS?

\bigskip

{\bf {\color{red}Problem 9:}} {\sf visgen} will fail if one 
attempts to ``observe'' too
large a field-of-view with too long a baseline; e.g.,:

ERROR: get$\underline~$patch() failed with code 2.
u,v: -42840.4,-1.62819e+06. udim,vdim: 2163.81,2093.3

The user needs some straightforward 
way of determining {\it a priori} what is the maximum FOV 
observable via MAPS with a given
antenna array.

{\bf Solution/Comments:} This problem may have been discussed by Shep
and Steven Tingay years ago. Unknown whether they thought about fixes.

\bigskip

{\bf {\color{red}Problem 10:}} Because one can only ``observe'' a small FOV when the baselines
become long, simulations involving realistic sky models are not
possible in the VLBI regime. (Simulated sky patches are so small
that they contain few or no sources at realistic flux density limits).

{\bf Solution/Comments:} Intrinsic limitation?


\bigskip

{\bf {\color{red}Problem 11:}} The program add$\underline~$noise.c
uses a system efficiency, $\eta$, to compute the theoretical noise for
each baseline. However, in line 630 of visgen.c, $\eta$ is hardwired to
be 1.0. 

{\bf Solution/Comments:} The user should be able to specify
$\eta$. This would likely need to be included in the ``array'' file. For VLBI, 
an extra switch to provide a choice between 1 and 2 bit sampling may 
also be desirable.


\bigskip

{\bf {\color{red}Problem 12:}} The program add$\underline~$noise.c
does not include the noise contribution from the total power received
at the antenna. 

{\bf Solution/Comments:} This part of the code needs completing.


\bigskip

{\bf {\color{red}Problem 13:}} obs$\underline~$.html quotes the
specification for the declination part of the pointing center as
PNT$\underline~$center$\underline~$DEC, as do the error messages
reported by read$\underline~$obs$\underline~$spec.c; however,
read$\underline~$obs$\underline~$spec.c actually expects this field to be 
PNT$\underline~$center$\underline~$Dec.

{\bf Solution/Comments:} Edit obs$\underline~$.html and fix messages
outputted by read$\underline~$obs$\underline~$spec.c so that the cases
match those needed in the obs$\underline~$ file. It would be better if
the obs$\underline~$spec file avoided mixed cases entirely, but
attempting to fix
this would probably cause additional problems. A better solution would
be that {\sf visgen} halts if it does not recognize a field in the obs
spec file instead of surreptitiously 
writing this information to the log file and continuing.

\bigskip

{\bf {\color{red}Problem 14:}} Coordinates of out-of-beam sources
written to the log file by oob.c do not carry enough significant
digits for the VLBI case. Furthermore, coordinates of ``OOB'' sources in
images produced from VLBI simulations can have errors of $\sim0.15''$. 

{\bf Solution/Comments:} Replacing ``\%g'' format statements with
``\%f'' in line 72 of oob.c would remedy first problem. Source of
latter problem unclear.

\bigskip

{\bf {\color{red}Problem 15:}} MAPS now accepts absolute
Cartesian coordinates (X,Y,Z) for station positions, but still requires
an ``array center'' specification even in this case. 
This creates confusion and potential for errors. 

{\bf Solution/Comments:}  Double-check this implementation. If absolute (X,Y,Z)
coordinates are specified, the program should ignore the specified
array center.

\bigskip

{\bf {\color{red}Problem 16:}} {\sf maps2uvfits} requires specification of an
``array center'', but does not seem to make use of this
information. 

{\bf Solution/Comments:}  If this information is not needed, eliminate
it from the command line inputs. If it is required, need to understand
why and also why results currently are unaffected by choice of values. 


\bigskip

{\bf {\color{red}Problem 17:}} 
Conversions between geodetic and Cartesian coordinates assume
a spherical Earth.

{\bf Solution/Comments:}  Use a more realistic Earth model.

\bigskip


{\bf {\color{red}Problem 18:}} Segmentation faults occur when there
are too many stations and/or too large a range in baselines.

{\bf Solution/Comments:} Problem has been identified as an array with
insufficient allocation.

\bigskip

{\bf {\color{red}Problem 19:}} The recipe for creating new
stn$\underline~$layout files is undocumented.

{\bf Solution/Comments:} The relevant information should be extracted
from the code and complied
into a help file.

\bigskip

{\bf {\color{red}Problem 20:}} The elimination of LOsim would leave
MAPS without a means of introducing non-point sources (e.g.,
elliptical Gaussians) into model data. 

{\bf Solution/Comments:} Introduce this functionality into
MAPS$\underline~$im2uv via an optional input source list. Could be
done in the $u$-$v$ plane. A flag could be implemented to allow use of
this feature even in cases where the user is not starting with a
pre-fabricated sky image.


\bigskip

{\bf {\color{red}Problem 21:}} If one uses an
externally-generated image as a sky model for a MAPS simulation and then
images the resulting model visibilities, 
the
coordinates in the resulting image will be mirror-reversed (i.e., 
the sign of CDELT1 in the FITS header will be incorrect, 
hence the coordinates of sources in the field will no
longer be correct). This 
does not occur if the sky model was generated by LOsim.

{\bf Solution/Comments:} According to Randall Wayth, this 
will require retrofitting two core MAPS
modules.

%\bigskip
%
%{\bf {\color{red}Problem 22:}} In cases where pointing and phase center
%are offset, incorrect ``pointing'' coordinates are written to the FITS header.
%
%{\bf Solution/Comments:} Problem in {\sf maps2uvfits}?
				   
\bigskip

{\bf {\color{red}Problem 22:}} Comparison between real and simulated
MERLIN data show that the PSF appears to have the wrong position angle.

{\bf Solution/Comments:} Times appear to be correct in final uvfits
file; source of discrepancy unclear.



\end{flushleft}

\end{document}