\documentclass[12pt,psfig]{article}
\usepackage[usenames,dvips]{color}

\textwidth=16.25cm
\textheight=22.7cm
\oddsidemargin=0.1cm
%\pagestyle{empty}
\topmargin=0.35in
\headsep=0.5cm


\newcommand{\as}[2]{\mbox{#1\farcs #2}}
\newcommand{\am}[2]{$#1^{'}\,\hspace{-1.7mm}.\hspace{.1mm}#2$}
\newcommand{\HI}{\mbox{H\,{\sc i}}}
\newcommand{\kms}{\mbox{\rm km\,s$^{-1}$}}
\newcommand{\SK}{$S^{3}$}
\def\lsim{~\rlap{$<$}{\lower 1.0ex\hbox{$\sim$}}}
\def\gsim{~\rlap{$>$}{\lower 1.0ex\hbox{$\sim$}}}

\begin{document}

\begin{flushright}
{\large SKA MEMO \#X}
\end{flushright}
\smallskip

%\centerline{\Large DRAFT -- DRAFT -- DRAFT}

\bigskip

\centerline{\large\bf MASSACHUSETTS INSTITUTE OF TECHNOLOGY}
\centerline{\large\bf HAYSTACK OBSERVATORY}
\smallskip
\centerline{\normalsize Westford, Massachusetts 01886}
\bigskip
\bigskip
\centerline{December 7, 2009}
\bigskip

\begin{flushright}
Telephone: 1-781-981-5407

Fax: 1-781-981-0590
\end{flushright}
\bigskip
\bigskip

\noindent{To: SKA Calibration and Processing Group (CPG)}
\bigskip

\noindent{From: Lynn D. Matthews}
\bigskip

%\noindent{\color{red}Subject: Importing Model Skies into MAPS}
\noindent{Subject: Importing Model Skies into MAPS}

\bigskip

\noindent This Memorandum describes procedures for importing model sky
images into the MIT
Array Performance Simulation (MAPS) package and for incorporating sky
models into simulated observations with an array of radio telescopes. 


\bigskip
\bigskip

\section{Background\protect\label{background}}
%
The Square Kilometer Array (SKA) is a proposed next-generation radio 
telescope that 
will operate at frequencies between $\sim$80~MHz to 30~GHz. Once 
fully operational, the SKA will  
be 50-100 times more sensitive than existing radio arrays. At 1.4~GHz,
this translates to a limiting flux density of $\sim$10~nJy
(1$\sigma$). 
This leap in sensitivity implies that observations, particularly at lower
frequencies, will be confusion limited. 
Moreover, currently favored SKA designs comprising large numbers of
small-diameter of dishes (so-called large-N, small-D arrays) will
naturally have large fields-of-view. Consequently, not only will the
confusion problem be enhanced, but instrumental and
atmospheric effects will vary significantly across the field. If not
handled properly, these effects will dramatically limit the achievable
dynamic range. New hardware 
developments (e.g., correlator FOV shaping;
Lonsdale et al. 2004) and new software (e.g., Cornwell 2007)
will be required to overcome these limitations. Testing and
optimizing these tools, as well as optimizing the overall design of
the SKA, demands the ability to realistically model the 
sky background seen during routine SKA observations. 

In a companion document (Matthews 2009), I
described how model sky backgrounds suitable for SKA simulations
can be generated using the
SKADS Simulated Skies ($S^{3}$) package developed at the University of
Oxford (Wilman
et al. 2008). Here I will describe how these sky models can be
incorporated into simulated radio frequency observations with a
user-defined ``virtual observatory'' using MAPS. 


\section{A Brief History and Overview of MAPS}
%
Development of the
MIT Array Performance Simulator (MAPS) began at
MIT Haystack Observatory in 2001 with the goal of providing a flexible
tool for the generation of simulated low frequency radio array observations
and for
testing new radio calibration and processing algorithms (Lonsdale
2001). The code is
written  in C. 

Since its inception, 
contributions to MAPS have been made by various groups.
In early 2004,
various modules of MAPS were shipped to Swinburne University in
Australia where a now-defunct ``members only'' interface to the program
still resides. More recently,
R. Wayth (now at Curtin University) and colleagues have
incorporated several new features into MAPS to make it suitable
for simulated observations with the Murchison
Wide-Field Array (MWA)\footnote{http://www.mwatelescope.org/}. 
These developments, as well as a more
extensive technical overview of the workings of MAPS and its full
suite of capabilities will be documented in Wayth et
al., in preparation. In the mean time, a bugzilla database
for  MAPS is currently being maintained 
at http://mwa-lfd.haystack.mit.edu/bugzilla, and
a brief summary of its
capabilities of MAPS can be found at
http://www.haystack.mit.edu/ast/arrays/maps/. 

MAPS comprises a collection of several different program modules.
Each is operated via a command line-driven interface, with user inputs
specified via command line switches and ascii templates 
(hereafter ``meta-files'').
A schematic illustrating how MAPS may be used for an
arbitrary simulation is shown in Figure~1.

MAPS was designed to accommodate detailed descriptions
of heterogeneous interferometric arrays and a variety of other highly
flexible user inputs. Some key features include:

\begin{itemize}

\item ability to input an arbitrary sky brightness distribution

\item option to include out-of-beam sources

\item user-specified array geometries (including placement and
  orientation of individual receptors)

\item station-based beam forming

\item variable station beams

\item observing specifications through an input template (RA and
  DEC; field-of-view; time
  and frequency resolution; bandwidth; channel width; correlator
  integration times; observation start and stop times)

\item option to include thermal noise

\item time-
and location-dependent ionospheric effects; modeling of large- and
small-scale ionospheric structure

\item fully polarized instrument response

\item ability to do all-sky simulations

\item ability to export simulated data into FITS format

\end{itemize}

\noindent MAPS simulations may be constrained to cover a small patch
of sky, but all-sky simulations are also possible. Results can be
exported in FITS format to allow further analysis through any number of
external analysis packages.

This Memo is not intended as a comprehensive Users Guide to MAPS, 
nor does it describe the inner
workings of the code (for some information on 
the latter, see Doeleman
2001a,b; Cappallo 2002; Wayth et al., in prep.); rather this Memo
is intended as a primer for 
novice users who wish to get started using MAPS and to become familiar
with some its basic capabilities. This is a working document, so I
also draw attention to current
bugs, quirks, and limitations, and point to areas where future development
work might be particularly valuable. 

%%%%%%%%%%%%%%%%% FIGURE 1 %%%%%%%%%%%%%%%%%%%%%%%%%%%%%%%%%%%%%%%%%%%%%%
\begin{figure}
\vspace{2.5in}
\special{psfile=/home/lmatthew/meetings/AASsummer09/poster/chunk1x.ps
  hscale=40 vscale=40 voffset=-5 hoffset=-60 angle=0}

\special{psfile=/home/lmatthew/meetings/AASsummer09/poster/chunk2x.ps
  hscale=40 vscale=40 voffset=-5 hoffset=220 angle=0}

\vspace{0.5cm}
\caption{A schematic depicting the process of
performing a realistic, simulated radio observation 
using the MAPS package. Text in  blue refers to MAPS modules
used to perform various steps. Green labels/arrows refer to
external packages that may be used for generating the initial input
sky model and for performing imaging or other analysis of the
simulated data. Red text  and arrows depict user-specified inputs that
may be included in the simulation.}
\end{figure}
%%%%%%%%%%%%%%%%%%%%%%%%%%%%%%%%%%%%%%%%%%%%%%%%%%%%%%%%%%%%%%%%%%%%%%%% 

\section{Installing MAPS}
%
MAPS remains under active development, and for this reason it is
not yet a publicly available code. The latest version of MAPS is maintained by
R. Wayth (r.wayth@curtin.edu.au), who may be contacted for
additional information on recent developments. 
The repository for the code resides on the
server {\sf mwa-lfd} at Haystack.  
Persons wishing to download MAPS will need to
have an account on this server and should contact R. Crowley 
(rcrowley@haystack.mit.edu) for further information. 

MAPS uses a ``subversion'' (svn) server for version control. 
For your initial
installation, you will need to ``checkout'' the code; thereafter,
updates can be made using the command ``svn update'' from your MAPS base 
directory. 

To perform the 
initial checkout, enter the following at the command line from the
directory where you would like to install MAPS:

\bigskip

\noindent {\tiny \%} svn checkout svn+ssh://mwa-lfd.haystack.mit.edu/svn/MAPS

\bigskip

\noindent This will create a subdirectory called {\sf MAPS} 
in the current directory. The entire code requires roughly a few
hundred megabytes of space. Several additional third-party packages
are also required before MAPS will fully function. A partial list of
these can be found in the README file that will be automatically deposited 
in the {\sf MAPS} directory. If
other necessary packages 
are missing, error messages that indicate these deficiencies 
will result when you attempt to run various MAPS modules.
The complete instructions for installing MAPS are also contained in
the README file.

Once MAPS is installed, the {\sf MAPS} directory will contain a number of
subdirectories:

\bigskip

\noindent {\sf array} contains a series of ascii files with suffix
``.txt'' that are used to describe antenna locations and properties

\bigskip

\noindent {\sf doc} contains a few pieces of (incomplete)
documentation (\S~\ref{help})

\bigskip

\noindent {\sf stn$\underline~$layout} contains a set of files (with suffix
``.layout'') that give the properties of the antennas or antenna tiles
to be used for a given simulation 

\bigskip

\noindent {\sf test} contains several subdirectories (e.g., 
{\sf test01}) that guide the user through sample MAPS
computations. In each case, these subdirectories include a ``notes'' file 
(e.g., notes$\underline~$test01.txt) that gives a description of the
computation and step-by-step instructions for its implementation, together with any meta-files
needed to execute the various steps. At the time of this writing, not
all of these test subdirectories were complete, and some of the
notes files contain minor typos. Nonetheless, working through 
a few of these examples is highly instructive 
for the novice MAPS
user.

\bigskip

\noindent {\sf LOsim}, {\sf maps2uvfits}, {\sf maps$\underline~$im2uv}, {\sf
  maps$\underline~$ionosphere$\underline~$generation}, 
{\sf maps$\underline~$makesky} contain source code
  and ``Make'' files for the respective MAPS programs

\section{Running MAPS}
%
\subsection{Starting MAPS}
%
The first step to any MAPS session 
is to ``source'' one of the required
set-up files (either {\sf sim$\underline~$setup.csh} or 
{\sf sim$\underline~$setup.sh}, depending on your shell) in the
{\sf MAPS} directory. Before running MAPS for the first time, 
you will need to edit whichever of these files
you will use in order to define some necessary paths. Paths that
will (or may) require editing in these set-up files 
are clearly labeled with comment lines. Once the paths are set, type:

\bigskip

\noindent {\tiny \%} source sim$\underline~$setup.csh (or source sim$\underline~$setup.sh)

\bigskip

\noindent and you should now be able to run MAPS programs from within
this window. Of course you can also set up your login file to do this
automatically in the future. 

\smallskip

A few other things that are useful to know before proceeding:

\begin{itemize}

\item All MAPS programs/commands are executed from the command line

\item Typing the name of most MAPS modules without any arguments
  (e.g., {\footnotesize\% {\sf maps2uvfits}}), will print to the terminal the
  calling sequence required to execute that program

\item Input to certain MAPS modules is supplied via ascii templates
  or ``meta-files''
  that must reside in your current directory (editing and use of these
  meta-files is discussed
  further below)
  
\item MAPS does
  not have any native image display
  capabilities. However, FITS images created using the MAPS program {\sf LOsim} can be
  viewed with any standard FITS viewer (e.g., ds9). 
   Visibility data created with MAPS will need to be converted to
  `UVFITS' format (using {\sf maps2uvfits}; see below) and then
  imported into other 
packages (e.g., AIPS, MIRIAD, CASA, IDL, etc.)
  for visualization or further processing.

\end{itemize}

\subsection{Where to go for Help\protect\label{help}}
%
Unfortunately, a comprehensive Users Guide to MAPS does not yet
exist. A few documents with some useful information are
contained in the {\sf doc} subdirectory:

\begin{itemize}

\item {\bf LOsim.ps} is an outdated
description of the MAPS {\sf LOsim} program. 
Ignore all information in
the ``Installation'' section of this manual, as it is obsolete; {\sf LOsim} is
now automatically installed with the rest of MAPS. The remainder of this
document, however, contains brief descriptions of the
meta-files needed to run LOsim.

\item {\bf manual.html} The most ``complete'' MAPS manual
  available, this web page contains a brief description of several
  MAPS modules, the commands for executing them, a list of
  input and configuration files needed, and a list of available command line
  switches

\item {\bf obs$\underline~$spec.html} describes each of the fields in
  the meta-files used to run the MAPS program 
{\sf visgen} (see \S~\ref{visgen})

\end{itemize}

\noindent Users who wish to better understand the inner
workings of MAPS may also wish to consult the series of memos
on the Haystack MAPS home page. These discuss
the simulation of
station beams within MAPS (Doeleman 2001a);   the intrinsic accuracy
of LOsim (Doeleman 2001b); and the integration in the $u$-$v$ plane of
the Fourier-transformed sky brightness distribution (Cappallo 2002).



\section{Some Step-by-Step Examples}
%
The best way to familiarize yourself with MAPS is by using it. As noted
above, the {\sf test} subdirectory contains instructions for various
sample computations. Here I supplement these 
with some additional 
examples specifically 
aimed at importing model sky images into MAPS and incorporating
these into mock observations. 

\subsection{Example 1: Importing a Model Sky Image into MAPS}
%

%%%%%%%%%%%%%%%%% FIGURE 2 %%%%%%%%%%%%%%%%%%%%%%%%%%%%%%%%%%%%%%%%%%%%%%
\begin{figure}
\vspace{2in}
\special{psfile=1.4_image_crop.ps
  hscale=60 vscale=60 voffset=190 hoffset=120 angle=-90}

\vspace{1cm}
\caption{A simulated $1^{\circ}\times1^{\circ}$ image of
a patch of radio sky at 1.4~GHz, 
produced using the $S^{3}$ software package (see
Matthews 2009). The  intensity scale is logarithmic. The simulation is
noise-free, so all features
correspond to ``real'' sources. For display purposes, the image has
been convolved with a $18''$ circular beam.}
\end{figure}

%%%%%%%%%%%%%%%%%%%%%%%%%%%%%%%%%%%%%%%%%%%%%%%%%%%%%%%%%%%%%%%%%%%%%%

Suppose you have  
in hand a FITS image of a $1^{\circ}\times1^{\circ}$ patch of 
sky at a center frequency of 
1.4~GHz called  ``1.4GHzmodelsky.fits'' (Figure~2), generated via the use
of the $S^{3}$ software package (see Matthews 2009), and this model image has
$6''$ pixels.

The  MAPS module {\sf MAPS$\underline~$im2uv} is able to read
a FITS format image, fast fourier transform (FFT) it, and write the
results in a binary format that the MAPS program {\sf visgen} will be
able to read (see visgen$\underline~$binary$\underline~$format.html in
the {\sf MAPS/doc} directory for information about this format). The
program {\sf visgen} is the ``heart'' of MAPS, and in the next step we
shall use it to generate
model visibilities. Ideally, we should be able to perform this FFT and
reformating 
with the following command:

\bigskip

\noindent{\tiny\%} MAPS$\underline~$im2uv -i 1.4GHzmodelsky.fits -o
SKAtst$\underline~$Visibility.dat -n 3.59e5

\bigskip

\noindent Here, the ``- i'' switch indicates the name of our input maps;
``-o'' precedes the desired output name (which must contain the suffix
{\sf $\underline~$Visibility.dat}). Unfortunately, {\sf 
MAPS$\underline~$im2uv} does not use any of the coordinate information
in the FITS header. This means that hereafter, MAPS no longer
``knows'' the intrinsic spatial scale of your image, the coordinate
epoch, or the brightness scale. We shall see that the loss of
this information can cause problems later on if we are not careful.
We can, however, adjust the units of the image using the ``-n''
switch; this switch
feeds MAPS$\underline~$im2uv a multiplicative constant to apply to the image
such that its output units will be Jy steradian$^{-1}$.
These are the only three switches currently accepted by MAPS$\underline~$im2uv;
all other options have now been depreciated.

Any image outputted from $S^{3}$ (regardless of whether it has been
convolved with a ``beam'') will have units of Jy pixel$^{-1}$.
In our example, the pixels are 6$''$ per size, implying a 
multiplicative constant of 3.59e5
will be required to maintain the correct units.\footnote{If instead your sky image was created with
the MAPS program {\sf LOsim} (see \S~\ref{losim}), the units are such that the
required scaling factor will be equal to the size of the image in
radians squared.} 
%In future versions of MAPS, it would be useful not to have to 
%calculate this constant ``by hand''.

If we now attempt the above command, we will immediately receive
an error message: 

\bigskip

\noindent{\tiny \%} input image is NOT 2 dimensional. wrong wrong wrong.

\bigskip

\noindent Our image has only two spatial dimensions, but the
program disputes this because it interprets the Frequency and Stokes
axes in the header as additional ``dimensions''. One
way to solve this problem is to import the original image into AIPS 
(using task {\sf FITLD}), transpose
the image so that Frequency (rather than RA) is the 
first axis (task {\sf TRANS}), and then use task {\sf XSUM} (with
{\sf OPCODE=`AVE'}) to
``collapse'' this axis, making it disappear from the header.
Repeat the same procedure for the Stokes axis, and you will be left
with an image that has only RA and DEC axes in its header. 

While we have the data inside AIPS, we can perform a
second crucial operation: padding the periphery of the image with 
zeros. This will 
mitigate aliasing
that can later introduce bogus sources (and/or their sidelobes) into
our model images. Padding can be  accomplished using the AIPS task {\sf PADIM},
and it is recommended to add a border equal to half the image
width (resulting in an image twice as large as the original). 
Following this step, we can
write the result back into FITS format (using {\sf FITTP}), to create
the file
{\sf 1.4GHzmodelsky2Dpad.fits}, and we are
ready to continue our MAPS simulation.

\bigskip

\noindent{\tiny\%} MAPS$\underline~$im2uv -i 1.4GHzmodelsky2Dpad.fits -o
SKAtst$\underline~$Visibility.dat -n 3.59e5

\bigskip

\noindent This time we should have successfully created an output
visibility file in ``visgen'' binary format. 
We are now ready to use our model sky
as part of a simulated observation (\S~\ref{example2}).

\subsection{Example 2: Using an Imported Model Sky Image as Part of a
Simulated Observation\protect\label{example2}}
%
Once our model sky image has been stripped of its 
Stokes and Frequency axes, padded
with zeros, and FFTed as described in the previous section, we may
now use it as part of a simulated MAPS observation. 

\subsubsection{Preparatory Steps}

Several preparatory steps are required before performing a simulated
MAPS 
observation:

\begin{enumerate}

\item
Enter the
coordinates of your observatory (in East Longitude and Latitude) 
into the {\sf sites.txt} file in the {\sf \$SIM/text} 
subdirectory. Several 
observatories are already listed in this file, but you may make
additions as
needed. If you wish to use any of the default entries, it is
recommended that you double-check 
to be sure that they correspond to
the latest and best-determined values. Each entry in this file consists of 
three columns; the first field is a character string of up to 12
letters that uniquely specifies a designation for the array. 
The next two columns are East Longitude
and Latitude, respectively of the array center. 
For locations less than 180$^{\circ}$ West
of Greenwich, a minus sign should precede the Longitude entry. Comment
lines in the {\sf sites.txt} file should contain an asterisk in the
first column. 

\item
Define the locations of the stations making up your radio telescope
array by creating an ``array'' 
file with suffix ``.txt'' in the {\sf \$SIM/array}
subdirectory. This directory will already contain several sample
files of this type. 
The ``old'' format for these files (which is still accepted; see, e.g.,
{\sf mwa$\underline~$32$\underline~$crossdipole$\underline~$gp.txt}) 
is to specify in the first two columns offsets (in
meters East and North, respectively) relative to the array center.
The third column may optionally contain a $z$ coordinate. The fourth column
(or third column if the $z$-term is null) should contain a string
indicating the type of antenna that is present at each
position. For a uniform array, this latter entry will be the same for all
antennas. The file that defines the corresponding antenna properties
resides in the {\sf stn$\underline~$layout} directory (see Step~3).

The ``new''  format for the array files contains 8
columns (e.g., Appendix~A). 
The first column contains a character string specifying a
unique name for each station (e.g., ``ant1'', ``ant2'',
etc.). Columns~2, 
3, and 4 contain the absolute $X$, $Y$, and $Z$ coordinates of each
station (in meters) relative to the center of the Earth\footnote{For
  an overview of this and other commonly used coordinate systems, see:
  http://www.colorado.edu/geography/gcraft/notes/coordsys/coordsys.html}. 
Column~5
should contain a string indicating the type of antenna (whose antenna
properties will be defined by a file in the {\sf stn$\underline~$layout}
directory; see Step~3). Columns~6 and 7 contain the lower and upper elevation
limits of the antenna in degrees, while Column~8 specifies the system
equivalent flux density (SEFD) for the antenna. See {\sf merlin4.txt}
for an additional example of an array file using the ``new'' format.

\item The {\sf \$SIM/stn$\underline~$layout} subdirectory
contains a series of ``station'' files, all with suffix
``.layout''. These files are used to define the properties of the antennas (or
multi-element stations) that make up your array. The base name for
the station file you wish to implement must match the string used in
Column~8 of your
``array'' file (or Column~4 if you are using the old format; see
Step~2). Suppose
we wish to build an  of 2-m parabolic dishes, and
in our ``array'' file we have designated these with the moniker {\sf
  dish$\underline~$2m$\underline~$unpol}. 
We now must create a file called {\sf
  dish$\underline~$2m$\underline~$unpol.layout} that consists of two
  lines. The first line defines the name of the antenna elements. In
  our case, this line would read, verbatim,:

\bigskip

\noindent NAME dish$\underline~$2m$\underline~$unpol

\bigskip

\noindent The second line consists of several columns:

\bigskip

\noindent 0.0 0.0 0.0 3 1.0 0.0 2.0

\bigskip

\noindent The first three
  columns define the
  $X$, $Y$, and $Z$ offset of the antenna from its nominal
  position (as defined in {\sf \$SIM/text/sites.txt}). 
   Generally these values will
  be zero unless there are multiple antennas comprising a single
  station. Column~4 defines the antenna type (3=ideal parabola with
  unpolarized receptor). Column~5 is the phase (assumed here to be 0), 
and finally Column~6
  is the antenna diameter in meters. Note that the number of columns
  and their meaning changes depending on  antenna type.  
%WHERE IS THE FULL RANGE OF OPTIONS DOCUMENTED?

 \end{enumerate}

\subsubsection{Preparing to Run {\sf visgen}\protect\label{visgen}}
After Steps~1-3 are completed, we are now ready to use
{\sf visgen} to perform a simulated observation. Running {\sf visgen}
without any arguments will list a complete summary of command line
switches. These are also described in {\sf \$SIM/doc/manual.html}.
In addition to the command line switches, {\sf visgen} must be fed a series
of inputs via an ascii-format meta-file known as an 
``obs spec'' file, whose format will look similar to
the following:

\smallskip
\begin{footnotesize}
\begin{flushleft}
FOV$\underline~$center$\underline~$RA = 00:00:00\\
FOV$\underline~$center$\underline~$Dec = -26:00:00\\
$//$ FOV in arcsec\\
FOV$\underline~$size$\underline~$RA = 7200.0\\
FOV$\underline~$size$\underline~$Dec = 7200.0\\
Corr$\underline~$int$\underline~$time = 1.0\\
Corr$\underline~$chan$\underline~$bw = 0.001\\
Time$\underline~$cells = 0\\
Freq$\underline~$cells = 0\\
\bigskip
$//$  Scan$\underline~$start = 2006:274:4:00:37\\
Scan$\underline~$start = GHA  -7.8237892\\
$//$ scan duration in seconds\\
  Scan$\underline~$duration = 3600.0\\
$//$ freq, bandwidth in MHz\\
  Channel = 1400:0.001\\
Endscan\\

\end{flushleft}
\end{footnotesize}
\smallskip

\noindent The obs spec file must reside in the directory where you plan to run
	  {\sf visgen}. Comment lines are indicated with double
	  backslashes.

As noted above, {\sf MAPS$\underline~$im2uv} does not pass along with
it any information from the header of the original sky
image when it creates the Fourier transform of this
image. 
Consequently, {\sf visgen} does not have any {\it a priori}
information about the coordinates of the
field center or the size of the field-of-view (FOV), and these values must
be explicitly passed to the program through the obs spec file. 
While the RA and DEC of your field center may seem arbitrary
for a simulated patch of sky, you must choose coordinates such that
your region will be observable from your observatory site at the start
time that you specify. Otherwise {\sf visgen} will (without warning) 
simply produce an
output of all zeros.
Also note that the FOV that you specify
in your obs spec file {\em must} match that of your original input
sky image, inclusive of any border of zero padding that has been added to its
periphery. If you specify a smaller region, {\sf
visgen} will not select a sub-region of your input sky image, it
will assume that whatever you specify is the intrinsic FOV of your 
sky model and scale
it accordingly!

The field ``Corr$\underline~$int$\underline~$time'' of the obs spec
file allows you to
specify your integration time in seconds; sometimes referred 
to as ``dump time'' or ``record
length''; this is not the total duration of your observation, but
rather the sampling and recording 
interval of the data.
``Corr$\underline~$chan$\underline~$bw'' specifies the channel width
in MHz. ``Time$\underline~$cells'' and ``Freq$\underline~$cells'' allow you to
subdivide the time and frequency cells into smaller increments to
improve computational accuracy; setting these values to zero will turn
off this option.

Two options are available for specifying the start time
(``Scan$\underline~$start''). The first option is to 
explicitly specify the time of your observation in Universal Time (UT)
(This option is commented out in the
sample file above). The required format is:

\bigskip

\noindent Scan$\underline~$start = YYYY:ddd:hh:mm:ss

\bigskip

\noindent where YYYY is the year, ddd is the day number (e.g.,
December 31 is day 365); hh is the UT hour, mm is the UT minute, and
ss is the UT second. Alternatively, you may specify a Greenwich 
Hour Angle (GHA;
sometimes called GST) as follows:

\bigskip
\noindent Scan$\underline~$start = GHA sH.DDDDDDD
\bigskip

\noindent The GHA refers to the hour angle your source
would have if measured at a given moment from Greenwich (irrespective of
whether or not the source is actually visible from your observing
location at this time). 
Here {\sf s} refers to the sign (+ or $-$), {\sf H}
indicates the hour, and {\sf .DDDDDDD} indicates decimal hours. 
Appendix~B contains a recipe for converting between LST and GHA.
The ``Scan$\underline~$duration'' field specifies the total duration
of your observation. 
Finally, the ``Channel'' field is specified as follows:

\bigskip

\noindent Channel = FFFF:B.BBB

\bigskip

\noindent where FFFF is the center frequency of your observation in
MHz and B.BBB is the total bandwidth in MHz. If the latter equals the
value of ``Corr$\underline~$chan$\underline~$bw'', your resulting
$u$-$v$ data set will comprise a single channel.

\subsubsection{Running {\sf visgen}}
%
Once your obs spec file is set up, you are ready to run {\sf
visgen}. In this example, we will use the obs spec file shown above
to perform a noise-free simulation. We will perform the observation
from a site in the Australian outback, whose latitude and
longitude are -26.62$^{\circ}$, 
E117.51$^{\circ}$. Our array configuration will comprise 49 2-m
parabolic antennas,
whose locations are specified according to the sample ``array'' file given
in Appendix~A. The maximum baseline is $\sim$2.5~km.

The calling sequence for {\sf visgen} would be as follows:

\bigskip

\noindent {\tiny \%}visgen -n SKAtst -s SKA$\underline~$SITE -A 
\$SIM/array/SKA$\underline~$core.txt 
-G SKAtst$\underline~$Visibility.dat -V 
obs$\underline~$spec$\underline~$SKAtst$\underline~$gha 
-N -m 0 $>$ visgen$\underline~$SKAtst.out

\bigskip

\noindent Here, the various command line switches have the following
meanings:

\begin{flushleft}
-n: name of the resulting model visibility file (a ``.vis'' extension
 will be added to the specified name) (required)

-s: site name as defined in \$SIM/text/sites.txt (required)

-A: the array configuration file (in \$SIM/array) 
defining the coordinates and types
 of the antennas or stations (required)

-V: name of the obs spec file (required)

-G:  name of the gridded $uv$ data produced by {\sf
 maps$\underline~$im2uv} (or {\sf LOsim}; see \S~\ref{losim}) (optional)

-N: instructs {\sf visgen} to do a noiseless simulation; otherwise
 Gaussian noise is added 

-m value: controls how verbose are the output messages; possible values
 are integers ranging from -2 (lots) to +2 little (optional)

\end{flushleft}

\noindent In additional to the switches used in our example, other
possible {\sf visgen} command line switches include the following:

\begin{flushleft}

-O name: include a list of user-specified point sources in the
 simulation (see \S~\ref{outofbeam}) 

-I name: turn on ionospheric modeling by naming an ionospheric model file

-i name of ionospheric settings configuration file (used only with ``-I'')

-S sefd: specify a global system equivalent flux density, sefd, for all
 stations when adding noise

-Z do not compute visibilities from an input $uv$ image (i.e., do not
 include a model sky background image from {\sf LOsim} or {\sf
 maps$\underline~$im2uv}) 

\end{flushleft}


\subsubsection{Converting {\sf visgen} Output into FITS Format using 
{\sf maps2uvfits}\protect\label{maps2uvfits}}
%
The output of {\sf visgen} will be a set of vector-averaged visibilities
written to a binary data file with extension ``.vis''. The formatting
of this file is further described in the following document:
\$SIM/doc/visgen$\underline~$binary$\underline~$format.html. This
output file cannot be read by standard interferometry
reduction packages such as AIPS or MIRIAD; therefore, 
to permit importation of your
model visibilities into these
packages for further analysis, you will therefore 
need to convert your {\sf visgen}
output into FITS format using the MAPS module {\sf maps2uvfits}. The
calling sequence for our example is:

\bigskip

\noindent \% {\sf maps2uvfits} SKAtst.vis SKAtst.uvfits -26.62 117.51 20.0
\$SIM/array/SKA$\underline~$core.txt

\bigskip

\noindent The file {\sf SKAtst.vis} is our output from {\sf visgen} while
SKAtst.uvfits is the name of the FITS file to be outputted. The three
numerical values that follow specify the latitude (in degrees),
longitude (in degrees), and elevation (in meters) of our
array. Finally, {\sf maps2uvfits} needs to be directed to to the  
array configuration file. The latitude, longitude, and array
configuration file are used in conjunction to produce an antenna (AN)
file to attach to the output FITS file. This is most critical if you
have used the ``old'' array configuration style format that specifies
latitude and longitude rather than X,Y,Z coordinates. However, even in
the latter case, this information is still used to define the array center.
This will be important if you wish to plot your visibilities (e.g.,
elevation versus time), although for imaging, just the $u$-$v$ coordinates and
visibility values are used.


\subsection{Imaging Your Model Visibilities}
%

%%%%%%%%%%%%%%%%% FIGURE 3 %%%%%%%%%%%%%%%%%%%%%%%%%%%%%%%%%%%%%%%%%%%%%%
\begin{figure}
\vspace{2in}
\special{psfile=1.4_mockimage_crop2.ps
  hscale=35 vscale=35 voffset=200 hoffset=120 angle=-90}

\caption{The sky image from Figure~1, as ``observed'' for
one hour with a
compact array of 49 2-m antennas and deconvolved using a standard
CLEAN deconvolution algorithm (11,000 iterations with no ``clean boxes''). 
The ``noise'' in this image arises primarily 
from limitations of an unrestricted CLEAN on such a crowded field. Note that
owing to a MAPS bug, the image is flipped east-west relative to the
original image in Figure~1.}
\end{figure}

%%%%%%%%%%%%%%%%%%%%%%%%%%%%%%%%%%%%%%%%%%%%%%%%%%%%%%%%%%%%%%%%%%%%%%

Model visibilities produced by {\sf visgen} and converted to FITS
format using {\sf maps2uvfits} can be imported into any standard
interferometry reduction package for further analysis. Because these
steps are specific to the package being used (AIPS, MIRIAD, etc.),
details about this process will not be described in this document. I
note however two important cautions that the user should be aware of
at this stage. First, after importing the data into an external
package, the user will likely need to sort the model visibilities in
order to arrange them in time-baseline (TB) order. Secondly, upon
imaging the model visibilities (Figure~3), you will find that the resulting image
is flipped east-west relative to your input image. Furthermore,
because the sign of CDELT1 is actually incorrect in the image header, this
cannot be rectified by a simple $x-y$ transpose. If the absolute
coordinates of the sources in the image are important, you will
to alter the image header to fix the erroneous
value of CDELT1.

%\subsection{Example 3: Generating a Simple Source Image using {\sf LOSim}}
%%
%For some applications, we may not wish to include a ``realistic'' sky
%background, but rather wish to generate an image with a custom-made
%distribution of sources whose positions, sizes, and brightnesses are 
%specified.
%MAPS currently
%provides two routes for doing this...with the module {\sf LOSim}
%or inclusion of an ``out-of-beam'' file with {\sf visgen}.
%
%Used with {\sf visgen}, the xxx switch 
%....allows the user to include a list of one or more point and/or extended
%(elliptical Gaussian) sources through an ``out-of-beam'' 

\subsection{Example 3: Adding Additional Sources to an Existing Simulation 
using {\sf visgen}\protect\label{outofbeam}}
%
For many types of simulations, the user may wish to add additional 
sources with specific positions and intensities 
to existing sky models. For example,
one may wish to test how a single bright point source located just 
outside the primary
beam will affect the ultimate dynamic range of the image, or one may 
wish to test
how well one can detect faint background galaxies in
an 
image where
several bright radio sources lie near the field center. In such cases, 
additional point
sources can be included in the simulation through the use of a 
so-called
``out-of-beam'' file when running {\sf visgen}. Note the sources
contained 
in this file (which must be named {\sf ooblist.txt})
need not lie outside the primary beam, but can lie anywhere within the 
field-of-view of interest.

A sample {\sf ooblist.txt} file is reproduced below:

\smallskip

\begin{flushleft}

\# format: RA (decimal hours), DEC (decimal degs), Stokes I,Q,U,V

0.177 -0.177 20.0 0.0 0.0 0.0

\end{flushleft}

\smallskip

\noindent In this example, we 
will introduce
into our simulation an unpolarized 
20~Jy point source  at a projected distance of $\sim$15$'$ from the field
center. Additional sources can be added by including additional lines.
At present, there is no way to include {\em extended} sources
through the use of an out-of-beam file, although these can be
incorporated into model images computed using the {\sf LOsim} module (\S~\ref{losim}). 

\subsection{Creating a Sky Model using LOsim\protect\label{losim}}
%
TBD
%\subsection{Other Worries and Problems}
%

\section{References}

\begin{itemize}
\item Cappallo, R. J. 2002, LOFAR Memo \#006,\\
http://www.haystack.mit.edu/ast/arrays/maps/006.pdf

\item  Cornwell, T. J. 2007, Astronomical Data Analysis
Software and Systems XVI, ASP Conf. Series 376, ed. R. A. Shaw et al.,
223

\item Doeleman, S. S. 2001a
LOFAR Memo \#004\\
http://www.haystack.mit.edu/ast/arrays/maps/004.pdf

\item Doeleman, S. S. 2001b
LOFAR Memo \#005,\\
http://www.haystack.mit.edu/ast/arrays/maps/005.pdf

\item Lonsdale, C. 2001, LOFAR Memo \#002,\\
http://www.haystack.mit.edu/ast/arrays/maps/002.pdf


\item Lonsdale, C. J., Doeleman, S. S., \& Oberoi, D. 2004,
Exp. Astron., 17, 345

\item Matthews, L. D. 2009, CPG memorandum

\item Wilman, R. J. et al. 2008, MNRAS, 388, 1335

\end{itemize}

\appendix
%
\section{Sample ``array configuration'' File Format\protect\label{arraysamp}}
%
The first five lines of the antenna array configuration file ({\sf \$SIM/array/SKA$\underline~$core.txt}) used for
Example~2 (\S~\ref{example2}) are reproduced below.
In this example, the ``new'' antenna format is used, whereby the 
antenna positions
are specified in X, Y, Z coordinates, in meters, relative to the center of the
Earth:

\smallskip
\begin{flushleft}


\#ANTENNA   X           Y            Z        DIAM

SKA0001 -2781116.248 5068884.493 -2680845.279  dish$\underline~$2m$\underline~$unpol  5  90  350

SKA0002 -2780931.295 5068861.322 -2681080.944  dish$\underline~$2m$\underline~$unpol  5  90  350

SKA0003 -2780362.166 5069180.303 -2681068.123  dish$\underline~$2m$\underline~$unpol  5  90  350

SKA0004 -2781009.073 5069176.864 -2680403.600  dish$\underline~$2m$\underline~$unpol  5  90  350

SKA0005 -2780720.855 5068957.308 -2681117.739  dish$\underline~$2m$\underline~$unpol  5  90  350

.

.

.
\end{flushleft}


\section{Conversion from LST to GHA\protect\label{GHA}}

To convert from Local
Sidereal Time (LST) to GHA, use the following recipe:

\begin{enumerate}

\item Convert the LST to decimal hours.

\item Convert the longitude difference between your observatory and
  Greenwich (which is at 0 deg longitude). Convert the result from 
 decimal degrees to hours of time
  time by dividing by 15.

\item If the observatory is at West longitude, add the result to the
  LST; if it is at East
  longitude, subtract. If the result is greater than 24, subtract 24;
  if the result is negative, add 24. The result is the GHA in hours. 

\end{enumerate}

\end{document}

TO PICK UP WHERE YOU LEFT OFF, 
SEE /home/lmatthew/SKA/MAPS/test/memosim/notes.txt

